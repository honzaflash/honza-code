\documentclass{article}

\usepackage[T1]{fontenc}
\usepackage[utf8]{inputenc}
\usepackage[a4paper, total={6in, 8in}]{geometry}
\usepackage{enumitem}

\setlist[description]{labelindent=25pt,style=multiline,
                      leftmargin=2.5cm} % popis přepínačů
\def\cmd{\fontfamily{lmtt}\selectfont\textbf} % styl pro příkazy

\title{Dokumentace}
\author{Jan Rychlý}
\date{20. 11. 2020}

\begin{document}

\maketitle
\section*{úvod}

\indent
\par Klicova\_slova je skript na extrahování klíčových slov
z textových souborů.
To dělá podle četnosti výskytu slov. Spočítá výskyt všech slov,
některé ignoruje a pak vypíše pouze ty nejfrekventovanější.
V základním běhu bez přepínačů zpracuje text na standardním vstupu
a pak vypíše slova s četností 95. percentilu a vyšší v pořadí od
nejfrekventovanějších po ty méně frekventovaná na standardní výstup.
Zároveň při tom ignoruje slova o délce menší než 3 a
jakákoliv slova ze souboru ./ignore\_list.txt pokud soubor existuje.
\par Pomocí přepínačů je možné určit minimální délku slov, jejich
minimální četnost, konkrétní percentil nebo danou četnost u každého
slova zobrazit. Lze tedy i např. získat četnost každého slova v
textu.
\par Je vhodné zadat konkrétní percentil podle rozsahu souboru.

\section*{spuštění}
synopse: \newline
{\cmd {./klicova\_slova.pl [-h] [-lN] [-cN] [-pN] [-v] [FILEPATH]}
} \newline
V základu skript čte ze standardního vstupu. Pokud zadáme
{\cmd FILEPATH} skript se pokusí
otevřít a číst soubor na této adrese.

\section*{přepínače}
\begin{description}
    \item[{\cmd {-lN}}] Zobraz puze slova o délce {\cmd N} nebo více
    \item[{\cmd {-cN}}] Zobraz pouze slova s {\cmd N} nebo více výskyty
    \item[{\cmd {-pN}}] Zobraz pouze slova s výskytem {\cmd N} 
            percentilu nebo více (přebíjí možnost {\cmd {-cN}}) 
    \item[{\cmd {-v}}] Zobraz u každého slova počet výskytů
    \item[{\cmd {-h}}] Vypiš zprávu o použití. 
\end{description}
      
\section*{ignorovací slovník}
\indent
\par Skript se pokaždé pokouší načíst slovník slov, které by měl
ignorovat, ze souboru ./ignore\_list.txt. Přesné shody s těmito
slovy se poté budou při analýze přeskakovat.
\par Každé slovo musí být v souboru samo na vlastním řádku
bez bílých znaků. 

\end{document}