\documentclass{article}

\usepackage[T1]{fontenc}
\usepackage[utf8]{inputenc}
\usepackage[a4paper, total={6in, 8in}]{geometry}


\def\cmd{\fontfamily{lmtt}\selectfont\textbf} % styl pro příkazy

\author{Jan Rychlý}
\date{25. 11. 2020}

\begin{document}

\section*{\huge Popis řešení}

\subsection*{Shrnutí}
\indent
\par Během řešení jsem používal Perl poprvé,
takže jsem musel mnoho vyhledávat. A i proto jsem se rozhodl
problém hledání klíčových slov řešit jednoduše možná by se dalo
říci až primitivně.

\par Vycházím z toho, že klíčoá slova by pravděpodobně měla
patřit mezi ta nejvíce používaná. Daný text tedy postupně procházím
po slovech a pomocí asociativního seznamu počítám výskyty
jednotlivých slov.

\par Nakonec iteruji seznamem klíčů (tedy slov) seřazených právě
podle jejich četnosti (tedy hodnoty pod klíčem v asoc. seznamu)
a vypíšu pouze několik nejfrekventovanějších.

\par V průběhu implementace jsem se rozhodl umožnit některé
parametry nastavit pomocí přepínačů, i když to nebylo v zadání.
Skript je tak mnohem užitečnější.

\subsection*{Detailnější popis}
\indent
\par Především na začátku skriptu zparsuji argumenty a nastavím
si podle toho proměnné - subprocedura {\cmd {parse\_args}}. Určitě
existují funkce, které zpracují argumenty podle standardu a které
by mi dovolili zpracovat sofistikovanější syntax přepínačů, ale
já jsem pro takovýto jednoduchý skript zůstal u "ručního"
regex porovnávání a klasické rovnosti.

\par Poté si načtu slovník slov, které chci ignorovat v
{\cmd{load\_ignores}}. To dělám tak, že se pokusím otevřít
určený soubor a pokud neexistuje tak se vrátím a nechám
{\cmd {\%ignore}} hash prázdný. Pokud se mi podaří soubor otevřít, 
tak z každéh řádku odstraním newline znak a použiji ho jako klíč
v {\cmd {\%ignore}}.

\par Následně přichází o něco méně elegantní větvení. Nepodařilo se
mi totiž zjistit jak file handly uložit do proměnné, a tak dochází
k nepěkné duplikaci kódu. Pokud totiž byla předána cesta k souboru,
nejdřív soubor otevřu (v režimu s UTF8 kódováním), pokud ne tak
pouze nastavím standardnímu vstupu UTF8 kódování, a pak v obou
větvích iteruji po řádcích, které rozděluji pomocí {\cmd split}
na slova. Pokud je slovo kratší než požadovaná délka (daná buď 
přepínačem nebo výchozí hodnotou), tak jej přeskočím. Jinak
slovo v malých písmenech použiji jako klíč a zvýším hodnotu.

\par Dále pokud byl použit přepínač {\cmd {-pN}} nebo
nebyl pužit přepínač {\cmd {-cN}} najdu
{\cmd N}. respektive výchozí 95. percentil výskytů a
tuto hodnotu použiji jako hranici výskytu.

\par Pak už jenom vypíši na standardní výstup všechna slova
- tedy klíče z {\cmd {\%key\_words}} - v pořadí od nejčetnějších
po méně četná, přičemž přeskakuji slova, která jsou
klíčem v {\cmd {\%ignore}}, a skončím vypisovat, pokud narazím
na slovo s menší hodnotou než je hranice.

\subsection*{Reflexe a komentář}
\indent
\par Největším nedostatkem je, že slova jako předložky a spojky se často
v textu opakují více než klíčová slova relevantní pro předmět textu.
To by jistě nemělo množinu podobných dokumentů zmenšit, ale potenciálně
by se tak mohly  zdát podobné dokumenty, které si nijak příbuzné
nejsou. Tento problém částečně řeší možnost přidat soubor se slovy,
která nechceme brát jako klíčová, ale pak samozřejmě záleží na
jeho rozsáhlosti.

\par Pak také úkol stěžuje skloňování slov, které v řešení ignoruji.
Jistě by se dalo pracovat detailněji s kořeny slov, ale jejich
algoritmycké hledání by bylo nesnadné.

\par Zároveň jsem skript testoval pouze pro krátké a středně
dlouhé texty, kde samozřejmě čím delší text je tím je větší šance
že mezi nejfrekventovanějšími slovy budou skutečně taková,
která mají nějaký vztah k předmětu textu.

\par Nebyl jsem si jistý přirozeným jazykem, který jsem měl použít.
Zadání požaduje, aby se soubor jmenoval česky a i uživatelský
manuál jsem napsal v češtině. Jsem ale více zvyklý psát názvy a
komentáře v angličtině a obecně je dle mého angličtina vhodnějším
jazykem v kontextu kódu. A jelikož jsem už v kódu měl angličtinu
použil jsem ji i pro výstup při chybových hláškách a přepínače help.
Uznávám tedy tuto jazykovou dualitu a jistě bych neměl problém
psát kód česky nebo naopak manuál napsat v angličtině.

\par A i v rámci stylu není kód 100\% konzistentní, protože jsem
se Perl učil za běhu, přebíral některé zvykolsti, které jsem
pochytil, a přirozeně je kombinoval se svými zvyky a preferencemi.
Stejně tady bych neměl problém se naučit stylu, který je preferovaný
v týmu. V jazycích, které znám lépe píši kód čistěji.

\par Na závěr bych zmínil pár možných rozšíření: přepínač na vypnutí
ignorovacího souboru, přepínač na zadání vlastní cesty k ignorovacímu
souboru a možnost ztotožnit více tavrů jednoho slova podle přidaného
slovníku přípon.

\end{document}